\begin{problem}{Labs}{standard input}{standard output}{1 second}{256 megabytes}

The trainers at December Course 2024 are overwhelmed by the number of signups this year. With only a small number of trainers available, it's common to see trainers moving between labs. TheRaptor wants to monitor each trainer's movements to ensure no one is slacking off.

All trainers start off not being in any labs. Trainers can move to a new lab $i$, and TheRaptor wants to know how many trainers are currently in lab $j$. Your task is to help TheRaptor answer these queries efficiently.

You will be given $q$ queries, which can be one of two types:

\begin{itemize}
\item $1\ s\ i$: Trainer $s$ leaves the lab they are currently in (if any) and moves to lab $i$. Trainer $s$ may move back to the same lab or to a different one (To go to the toilet lol).

\item $2\ j$: Output the number of trainers currently in lab $j$.
\end{itemize}

\InputFile
The first line contains a single integer $q$, the number of queries.

This is followed by $q$ lines, each describing a query. Each query is in one of the forms:

\begin{itemize}
\item $1\ s\ i$

\item $2\ j$
\end{itemize}

as described above. $s$ is a string of lowercase English letters representing the trainer's name. $i$ and $j$ are integers representing lab indices.

\OutputFile
For each type $2$ query, output the number of trainers currently in lab $j$, each on a new line.

\Scoring
For all testcases, it is guaranteed that
 
\begin{itemize}
\item $2 \leq q \leq 10^6$

\item Each trainer's name, $s$, is a string consisting of at most 10 lowercase alphabetic characters.

\item $1 \leq i, j \leq 10^5$
\end{itemize}


\begin{center}
\begin{tabular}{|c|c|c|} 
\hline
\textbf{Subtask} & \textbf{Score} & \textbf{Additional constraints} \\
\hline
1 & 30 & Each trainer will move at most once \\ 
\hline
2 & 15 & $q \leq 10^3$ \\
\hline
2 & 40 & $q \leq 10^5$ \\ 
\hline
3 & 15 & $q \leq 10^6$ \\ 
\hline
\end{tabular}
\end{center}


\Example

\begin{example}
\exmpfile{example.01}{example.01.a}%
\end{example}

\Note
For the sample testcase:

After the first $2$ queries, ``alice'' and ``bob'' moves to lab $2$ and $3$ respectively.

The third query asks for the number of trainers in lab $2$. The result is $1$ (``alice'').

After the fourth query, ``alice'' moves to lab $3$.

The fifth query asks for the number of trainers in lab $3$. The result is $2$ (``bob'', ``alice'').

After the sixth query, ``charlie'' moves to lab 3.

The seventh query asks for the number of trainers in lab 3. The result is $3$ (``bob'', ``alice'', ``charlie'').

\end{problem}

